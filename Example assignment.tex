\documentclass[]{article}
\usepackage[utf8]{inputenc}
\usepackage[T1]{fontenc}
\usepackage{lipsum} %
\usepackage{linegoal}
\usepackage{tabularx}
\usepackage{wrapfig}
\usepackage{graphicx}
\graphicspath{ {./images/} }
\usepackage[T1]{fontenc}
\usepackage{titling}
\setlength{\droptitle}{-12em}
\newcommand{\subtitle}[1]{%
\posttitle{%
\par\end{center}
\begin{center}\LARGE#1\end{center}
\vskip0.5em}%
}


%opening
\title{\mbox{\textbf{Øvelsesopgaver i rumlige flader og legemer}}}
\author{Mathias 1.V}

\begin{document}
\maketitle
\thispagestyle{empty}
\section*{Opgave 1}
\textnormal{a)}
\hspace{1.1em}\parbox[t]{\linegoal}{En bordplade har følgende mål: Bredde = 45 cm, længde = 63 cm og tykkelse = 22 mm.\\}
\newline
Du skal bestemme bordpladens totale overfladeareal i $cm^{2}$ og $m^{2}$
\[22\:mm\;=\;2,2\:cm\]
\begin{center}
$2\times45\times63+2\times(45+63)\times2,2$\;\;=\;\;6145,2\:$cm^{2}$\:eller\:0,61452\:$m^{2}$
\end{center}
\section*{Opgave 2}
Et vinkelhus har udseende og mål som vist i figuren 6,8. Taget skal males.
\begin{center}
\includegraphics{image.jpg}
\end{center}
\begin{flushright}
Figur 6,8
\end{flushright}
a) \hspace {1.1em} Du skal bestemme arealet af den samlede tagflade. \\
\[Pythagoras: A2 = 42 + 42\]
\[(5,657\times12 + 5,657\times10)\times2 = 248.908\]
\[5,657\times4 = 22,628\]
\[5,657\times8 = 45,256\]
\vspace {0em}
\[248,908 + 22,628 + 45,256 = 316.792\]
\section*{Opgave 3}
\textnormal{Et bassin har tvær- og længdesnit som vist på figur 7,3.}
\newline
\phantom{}\hspace{2em}Målene er i meter.
\newline
\begin{center}
\includegraphics{image2.jpg}
\end{center}
Jeg starter med at opdele figuren i 2, en kasse og en trekantet prisme.
\newline
\newline
Kassen:
\vspace{-1.2em}
\[2\:\times\:2\:\times\:3\:=\:12\]
Prismen: 
\vspace{-2.2em}
\[Da\:siden\:er\:45°,\:og\:begge\:sidelængder\:er\:2,\:så\:må\:trekantens\]
\[længde\:også\:være\:ens\:med\:sidelængderne.\:Derefter\:udregnede\:jeg,\]
\[(2\:\times\:2\:\times\:3)/2 = 6\]
\begin{center}
\vspace{-0.5em}
Så lagde jeg resultaterne sammen, 12 + 6 = 18 m3
\end{center}
\end{document}